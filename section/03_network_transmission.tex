\chapter{ネットワークを活用した映像伝送}
\label{chap:network-transmission}

\section{仮説}
映像のIP伝送については、古くから研究がなされており、拠点間での映像伝送のため部分的な製品などが存在している。
しかし、この論文で扱っているIP伝送では、拠点間だけにとどまらず、カメラやスイッチャー、ディスプレイまでもをIP化するというテーマである。
そのため、実際に制作の現場にIP伝送を普及させた際に、現在の伝送方法の課題が解決でき、IP伝送を利用することができるのかについて検証する。

\subsection{Ethernetを活用するメリット}

\ref{chap:introduction}章で、述べた通り、ネットワークを活用することによって活かせるメリットは以下の3つである。

\begin{itemize}
  \item 1本のケーブルで複数や双方向の映像が可能
  \item 伝送スピードの向上
  \item コストダウン
\end{itemize}

また、映像伝送における重要なポイントは以下の3つである。

\begin{itemize}
  \item 画質、音質の劣化がない
  \item 伝送遅延を一定以下にする必要がある
  \item 安定性がある
\end{itemize}

また、IP伝送のメリットは以下の点である。

\begin{itemize}
  \item ユニキャスト、ブロードキャストが行える
\end{itemize}

これらの点において評価を行う。

\section{目的}

映像の制作の現場では、符号化・複合などによる映像の遅延を抑えることや、画質をそのまま伝送することがしばしば求められます。
このような背景から、4K映像を非圧縮のままIP伝送する技術の実証実験と、現場においてIP伝送を利用する際に問題となる点を洗い出します。?

% \section{構成}

% \section{関連研究}
