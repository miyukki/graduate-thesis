\chapter{NG-HDMI-TS}
\label{chap:network-transmission}

% \section{仮説}
映像のIP伝送については既に多くの先行研究があり、映像を拠点間などで伝送するための製品なども存在している。

しかし、本論文では、拠点間のIP伝送だけにとどまらず、拠点内の設備までもをIP伝送することをテーマとしている。
拠点内の設備として、カメラやスイッチャー、ディスプレイなどの拠点内の設備までもをIP伝送で行う、Video over IP化にするというテーマである。

そのため、実際に制作の現場にIP伝送を普及させた際に、現在の伝送方法の課題が解決でき、IP伝送を利用することができるのかについて検証する。

% \subsection{Ethernetを活用するメリット}

\ref{chap:introduction}章で、述べた通り、ネットワークを活用することによって活かせるメリットは以下の3つである。

\begin{itemize}
  \item 1本のケーブルで複数や双方向の映像が可能
  \item 伝送スピードの向上
  \item コストダウン
\end{itemize}

しかし、Ethernetを利用するためにはデメリットがある

輻輳
導入コストの

また、映像伝送における重要なポイントは以下の3つである。

\begin{itemize}
  \item 画質、音質の劣化がない
  \item 伝送遅延を一定以下にする必要がある
  \item 安定性がある
\end{itemize}

また、IP伝送のメリットは以下の点である。

\begin{itemize}
  \item ユニキャスト、ブロードキャストが行える
\end{itemize}

これらの点において評価を行う。

\section{目的}

映像の制作の現場では、符号化・複合などによる映像の遅延を抑えることや、画質をそのまま伝送することがしばしば求められます。
このような背景から、4K映像を非圧縮のままIP伝送する技術の実証実験と、現場においてIP伝送を利用する際に問題となる点を洗い出します。?

% \section{構成}
