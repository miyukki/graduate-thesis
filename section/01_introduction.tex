\chapter{序論}
\label{chap:introduction}

\section{本論文の背景}
近年、現行のハイビジョン放送を超える超高繊細な画質による、4K・8K映像の普及が世界中で加速している。また、日本でも東京2020オリンピック・パラリンピックに向け、総務省が4K・8K映像の普及を後押しをしている。

現在の放送業界では、同軸ケーブルを使用するSDIと呼ばれる伝送規格で映像を伝送する事が一般的である。また、ハイビジョン放送の制作では、1080iと呼ばれる有効走査線数1080本のインターレース方式が一般的である。
現行のハイビジョンと比較すると、4K映像の帯域は約8倍にもなり、伝送方法とそのコストが課題となっている。

SDIでは4K映像の伝送を目的として、SMPTE ST-2081 および SMPTE ST-2082により、6G-SDIと12G-SDIが定められている。
しかし、現行のハイビジョンで使用されるHD-SDIと比較すると、高周波による減衰を少なくするため、より太くより短い同軸ケーブルを使用しなければならず、現場での扱いづらさが課題である。

そのため、近年では、映像機器と比べて比較的安価なネットワークリソースを活用して伝送する、Video over IPが進んでいる。
SDIの伝送と比較して、IP伝送には次のようなメリットがある\cite{kodera-interbee2015}。

\begin{itemize}
  \item 1本のケーブルで複数や双方向の映像が可能\mbox{}\\
    同軸ケーブルとは異なり、双方向での通信が可能なため、双方向の映像の伝送が1本のケーブルで可能である。
    SDIでは、単一の映像伝送を目的としているが、IP伝送では帯域が許す限り複数の映像が伝送可能である。
    また、IPであるため、他のソフトウェアのデータなどの付加情報も可能である。
  \item 伝送スピードの向上\mbox{}\\
    SDIでは、一般的に銅線を使用した同軸ケーブルを使うため、高周波を扱う場合に限界がある。
    しかし、IP伝送で用いられる光ケーブルでは、より高周波を扱うことが可能である。
  \item コストダウン\mbox{}\\
    ネットワーク機器は、映像業界よりも先に10Gや400Gなどの帯域に対応し、高速化が進んでいる。
    そのため、それらのリソースを活用することで映像機器に比べて、コストを飛躍的に抑えることが可能である。
\end{itemize}

\section{本論文が着目する課題}
Video over IPにおける伝送規格は、SMPTE 2022とソニーのNMIが主流となっている\cite{kodera-interbee2016}が、その他にも多くの規格が提唱され、市場では

\section{本論文の目的}
本論文では、

ネットワークを
\cite{ntt-jgn-4k}

着目し、
IPを利用するメリットが得られるはずである。

\section{本論文の構成}
本論文における以降の構成は次のとおりである。

\ref{chap:video-transmission}章では、本論文を理解するための前提となる、汎用的な映像伝送システムについての解説をする。色空間と帯域の関係などにも触れる。
\ref{chap:network-transmission}章では、本論文の要となるネットワークを活用した映像伝送システムについて、本論文での実装との違いを踏まえ紹介する。
\ref{chap:implementation}章では、ネットワークを活用した映像伝送システムを、ソフトウェアとハードウェアで設計、実装したことについて解説をする。
\ref{chap:evaluation}章では、\ref{chap:implementation}章で実装した映像伝送システムを、既存のIP伝送装置と比較をし、評価を行い、その結果について考察する。
\ref{chap:conclusion}章では、本論文のまとめと今後の展望についてまとめる。

また、付録\ref{chap:orf2015}では、ORF2015での100Gbps回線を使用した映像伝送と遠隔スイッチングの実証実験についてまとめる。
付録\ref{chap:orf2016}では、ORF2016での実証実験についてまとめる。
