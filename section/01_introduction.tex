\chapter{序論}
\label{chap:introduction}

\section{本論文の背景}
近年、現行のハイビジョン放送を超える超高繊細な画質による、4K・8K映像の普及が世界中で加速している。また、日本でも東京2020オリンピック・パラリンピックに向け、総務省が4K・8K映像を後押しをしている。

現在の放送業界では、同軸ケーブルを使用するSDIという規格で映像を伝送する事が一般的である。また、ハイビジョン放送の制作では、1080iと呼ばれる有効走査線数1080本のインターレース方式が一般的である。
現行のハイビジョンと比較すると、4K映像の帯域は約8倍にもなり、伝送方法とそのコストが課題となっている。

SDIでは4K映像の伝送を目的として、SMPTE ST-2081 および SMPTE ST-2082により、6G-SDIと12G-SDIが定められている。
しかし、現行のハイビジョンで使用されるHD-SDIと比較すると、高周波による減衰を少なくするため、より太くより短い同軸ケーブルを使用しなければならず、現場での扱いづらさが課題である。

そのため、近年では、映像機器と比べて比較的安価なネットワークリソースを活用して伝送する、Video over IP化が進んでいる。
SDIの伝送と比較して、IP伝送には次のようなメリットがある。

\begin{itemize}
  \item 1本のケーブルで複数や双方向の映像が可能\mbox{}\\
    同軸ケーブルとは異なり、双方向での通信が可能なため、双方向の映像の伝送が1本のケーブルで可能である。
    SDIでは、単一の映像伝送を目的としているが、IP伝送では帯域が許す限り複数の映像が伝送可能である。
    また、IPであるため、他のソフトウェアのデータなどの付加情報も可能である。
  \item 伝送スピードの向上\mbox{}\\
    SDIでは、一般的に銅線を使用した同軸ケーブルを使うため、高周波を扱う場合に限界がある。
    IP伝送で用いられるケーブルでは、帯域では10Gや100G、さらには400Gなど高周波を扱うことが可能である。
  \item コストダウン\mbox{}\\
    ネットワーク機器では、すでに映像業界よりも先に10Gや400Gなど、高速化が進んでいるため、それらのリソースを活用することで映像機器に比べコストを飛躍的に抑えることが可能である。
\end{itemize}

\section{本論文が着目する課題}
Video over IPにおける、伝送規格は、SMPTE 2022とソニーのNMIが主流となっているが、その他にも多くの企画が提唱され、
IPを利用するメリットが得られるはずである。

\section{本論文の構成}
本論文における以降の構成は次のとおりである。

\ref{chap:video-transmission}章では、
\ref{chap:network-transmission}章では、
\ref{chap:implementation}章では、
\ref{chap:evaluation}章では、
\ref{chap:conclusion}章では、

\section{本論文の目的}
本研究では、〜

% \chapter{序論}
% \label{chap:introduction}
%
% 論文は序論のようなもので始める。タイトルは序論でも序言でもはじめにでもいいけど、『序論』で始めたら『結論』で終わり、『序言』で始めたら『結言』で終わるようにする。『はじめに』なら『おわりに』で終わる。『序論』で始まって『おわりに』でおわるとか、そういうちぐはぐなのはだめ。
%
% ここでは序論として書く。序論では、研究の背景やら目的やらを書くのが普通。今はテンプレートの説明なので、大して書くことは無い。
%
%
% \section{背景}
%
% ここではこのテンプレートのオリジナルの作者である @kurokobo の書いたもの\cite{kurokobo10}を引用したい。
%
% \begin{quotation}
% ぼくは別に\LaTeX に明るいわけではなくて、この研究室に所属してから初めて触った程度。四年生になってぼく自身が卒業論文を書くことになって、先生は\LaTeX を推奨していたんだけど、テンプレートありますかって聞いたら特にないから作ってほしいとのことだったので、じゃあ作りますよ、という流れ。ぼく自身が使いやすいように、自分が使いながらいろいろ改良をして、こうして公開している。
%
% 作成にあたっては、先輩方の卒業論文や主にぐーぐる先生を活用したインターネット上の情報を参考にした。
%
% ただ、卒業論文の体裁は、それぞれの研究室の文化や、担当の指導教員のこだわりも強く影響することも事実。このテンプレートは、『ぼくが所属していた研究室』という、ごくごく限定的でローカルな仕様に沿ったフォーマット――より正確に言えば『ぼくが所属していた研究室ではNGではなかった』フォーマット――というだけのもの。そのあたり、承知の上で使ってほしい。
%
% 他の研究室で使う場合は、指導教員の許可を仰ぐほうが確実。
% \end{quotation}
%
% 筆者はこの卒業論文用のテンプレートを大学院ガイドに例示されている体裁\cite{mag_guide12}に沿うように改造した。これは論文の形式で言えばもっと後ろに書いてあるべきことなのかもしれない。
%
% \section{本文書の構成}
%
% 第1章の最後は、文書全体の構成を大まかに書くとよいらしい。
%
% 第\ref{chap:introduction}章では本テンプレートの概要みたいなものを書いた。第\ref{chap:howto}章では、本テンプレートの使い方を説明する。第\ref{chap:latex}章で図表や数式の挿入など代表的な\LaTeX コマンドを解説する。第\ref{chap:conclusion}章では、『序論』で始めたら『結論』で終われと書いた手前書かざるを得ないので、なにか結論らしいことを書く。付録として、テンプレートのサンプルになるように無理矢理ゴミを添付する。
