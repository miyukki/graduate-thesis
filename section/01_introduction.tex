\chapter{序論}
\label{chap:introduction}

\section{本論文の背景}
近年、現行のハイビジョン放送を超える高解像度な画質による、4K・8K映像の普及が世界中で加速している。また、日本でも東京2020オリンピック・パラリンピックに向け、総務省が4K・8K映像の普及を後押しをしている。

現在の放送業界では、同軸ケーブルを使用するSDIと呼ばれる伝送規格で映像を伝送する事が一般的である。また、ハイビジョン放送の制作では、1080iと呼ばれる有効走査線数1080本のインターレース方式が一般的である。
現行のハイビジョンと比較すると、4K映像の帯域は約8倍にもなり、伝送方法とそのコストが課題となっている。

SDIでは4K映像の伝送を目的として、SMPTE ST-2081\cite{smpte-st-2081}、および、SMPTE ST-2082\cite{smpte-st-2082}により、6G-SDIと12G-SDIが定められている。
しかし、現行のハイビジョンで使用されるHD-SDIと比較すると、高周波による減衰を少なくするために、より太く短い同軸ケーブルを使用しなければならず、現場での扱いづらさが課題である。

多くの放送局のスタジオなどで使用しているカメラでは、カメラ本体とCCU(カメラコントロールユニット)を光ファイバーで接続することが主流となっている。
しかし、その接続方式は機器やメーカーごとに独自のものであり、光ファイバーのメリットを活用できていない。

そのため近年では、映像機器と比べて比較的安価なネットワークリソースを活用して、映像をIPパケットにして伝送する、Video over IP化が進んでいる。

SDIによる伝送と比較して、IP伝送には次のようなメリットがある\cite{kodera-interbee2015}。
\begin{itemize}
  \item 1本のケーブルで複数や双方向の映像伝送が可能\mbox{}\\
    同軸ケーブルとは異なり、双方向の映像伝送が1本のケーブルで可能である。
    SDIでは、単一の映像伝送を目的としているが、IP伝送では帯域が許す限り複数の映像や他のソフトウェアのデータなどの付加情報の伝送が可能である。
  \item 伝送スピードの向上\mbox{}\\
    SDIでは、一般的に銅線を使用した同軸ケーブルを使うため、高周波を扱う場合に物理的な限界がある。
    しかし、IP伝送で用いられる光ケーブルでは、より高周波を扱うことが可能である。
  \item コストダウン\mbox{}\\
    ネットワーク機器は、映像制作現場よりも先に40Gbpsや100Gbpsなどの帯域に対応し、伝送スピードの高速化が進んでいる。
    そのため、それらのリソースを活用することで映像機器に比べて、コストを飛躍的に抑えることが可能である。
\end{itemize}

\newpage
また、IP伝送の規格であるソニーのNMI\cite{sony-nmi}では、次のようなメリットがある。
\begin{itemize}
  \item ライブシステムとファイルベースシステムを統合\mbox{}\\
    リアルタイムなライブ映像の伝送だけではなく、ファイルベースのシステムと統合することが可能である。
  \item システムの柔軟性\mbox{}\\
    ネットワーク機器を利用しているため、帯域が許す限りHDから4Kへの移行もスムーズに行なうことができる。
  \item 経路の多重化\mbox{}\\
    ルーティングシステムに障害が発生した場合、今まではケーブルの差し替えやパッチなどで対応することが一般的であったが、
    IP網での伝送経路を変更することにより、インフラ設備に対しての可用性を高めることが可能である。
\end{itemize}

このようにVideo over IP化によるメリットは大きく、映像制作現場において、Video over IP化が今後進んでいくことは明確である。

\section{本論文の目的}
% <<<<<<< HEAD
映像のIP伝送については既に多くの先行研究があり、映像を拠点間などで伝送するための製品なども存在している。
しかし、本論文では拠点間のIP伝送だけにとどまらず、拠点内の設備までもをIP伝送する、Video over IPに着目する。
実際の映像制作現場でIPによる映像伝送を普及させた際に、現在の拠点間のIP伝送が抱える課題を洗い出し、拠点内でも快適にIP伝送を利用することができる要点をまとめ、それが可能であるかについて検証する。

% Video over IPは拠点間の接続としてIP伝送装置が主に使われてきたが、本論文では、映像制作現場などの拠点内におけるIP伝送に着目する。
% まず、映像制作現場においてIP映像伝送に必要な要件をまとめる。次に、評価をおこなうため、ハードウェアによる4K IP伝送装置のNG-HDMI-TSを実装し、実験を行う。
% NG-HDMI-TSが拠点内の映像制作現場で必要となる要件を満たすことができるかどうかを検証する。
% 本論文では、Video over IP化が進んでいる映像制作現場において、これまでの拠点間におけるIP伝送装置ではなく、拠点内でのIP伝送装置に着目する。
% 拠点内でのIP伝送装置を映像伝送の手段として使用するにあたって、映像制作現場において必要な要件をまとめる。
% IP映像伝送の実装として、ハードウェアによる4K IP伝送装置のNG-HDMI-TSを実装し、実験を行う。
% 映像制作現場において、IPによる映像伝送が拠点内の映像制作現場で必要となる要件を満たすことができるかどうかを検証する。
% =======
% 本論文では、Video over IP化が進んでいる映像制作現場において、これまでの拠点間におけるIP伝送装置ではなく、拠点内でのIP伝送装置に着目する。
% 拠点内でのIP伝送装置を映像伝送の手段として使用するにあたって、映像制作現場において必要な要件をまとめる。
% IP映像伝送の実装として、ハードウェアによる4K IP伝送装置のNG-HDMI-TSを実装し、実験を行う。
% 映像制作現場において、IPによる映像伝送が拠点内の映像制作現場で必要となる要件を満たすことができるかどうかを検証する。
% >>>>>>> 34230b39d5f8daf3e2974656816ada4510c4b7cf
% また、映像伝送として映像の遅延などの許容値を計測し、評価する。
% また、それらによってまとまった要件を元に
% 4K解像度の映像のIP伝送装置が実用的なものであるかを、実装、実験、評価を行う。
% 必要な要件をまとめ、
% また、必要な要件を
% 映像制作現場における\\高解像度映像IP伝送システムNG-HDMI-TSの実装と評価
% 映像制作現場におけるIP伝送装置に着目する。
% 製品
% 本論文では、
% ネットワークを
% \cite{ntt-jgn-4k}
% 着目し、
% IPを利用するメリットが得られるはずである。

\section{本論文の構成}
本論文における以降の構成は次のとおりである。

% ネットワークを活用した映像伝送システムを、NG-HDMI-TS について、本論文での実装との違いを踏まえ紹介する。
% 伝送装置 伝送システム
\ref{chap:video-production}章では、本論文を理解するための前提となる、映像制作現場における構成と映像機器の要素技術についての解説をする。色空間と帯域の関係などにも触れる。
% \ref{chap:video-production}章では、本論文の要となる、映像制作現場におけるIP伝送装置について、その要件をまとめる。また、要件を元に実装したIP伝送装置、NG-HDMI-TSについての解説をする。
\ref{chap:software-experimentation}章では、IP伝送装置を、NG-HDMI-TSとは異なるソフトウェアで実装したことについて、その実装内容と評価、問題点をまとめる。
\ref{chap:implementation}章では、IP伝送装置のNG-HDMI-TSを実装したことについて、その実装内容をまとめる。
\ref{chap:evaluation}章では、\ref{chap:implementation}章で実装したNG-HDMI-TSを、\ref{chap:software-experimentation}章で実装したソフトウェアのIP伝送装置や既存のIP伝送装置と比較をし、評価を行い、その結果について考察する。
\ref{chap:conclusion}章では、本論文のまとめと今後の展望についてまとめる。

% また、付録\ref{chap:orf2015}では、ORF2015での100Gbps回線を使用した映像伝送と遠隔スイッチングの実証実験についてまとめる。
% 付録\ref{chap:orf2016}では、ORF2016での実証実験についてまとめる。
