\chapter{映像伝送システム}
\label{chap:video-transmission}

本章では、映像伝送システムを構成する技術要素の仕組みについて解説する。
% それぞれ、○○について述べる

\section{ビデオカメラ}
\label{sec:camera}


\section{ディスプレイ}
\label{sec:display}

\ref{sec:interface}


\section{インターレース}
\label{sec:interlace}

画像伝送において、データレートを増やさずに描画回数を増やす技術である。
この方式の特徴は、「人間の視野は動くものの細部を捉えられない」という性質に基づいている。
飛び越し走査とも呼ばれ、奇数番目の走査線を先に送り、偶数番目の走査線をその後に送る。
アナログテレビによく使われ
また、現在でも走査線を1ラインに割り当てられたものとして、ハイビジョン放送でも利用されている。


\section{色空間と色深度}
\label{sec:colorspace}

一般的に液晶ディスプレイでは、1ピクセルを赤、緑、青、すなわちRGBの3つの色信号で表現する。
多くのPCやゲーム機の出力ではRGBの色空間が使われ、RGBそれぞれ8bit、1ピクセルあたり24bitで表現する。
1ピクセルあたりを表現するビット数を色深度といい、色解像度、色分解能とも言われる。
24bitの色深度では、16,777,216色を表現することができる。

一方、ビデオカメラでは、輝度信号Yと、2つの色差信号を使って表現される色空間であるYUVが使われるが多い。[要出典]
この方式の特徴は、「人間の目は明るさの変化には敏感だが、色の変化には鈍感である」という性質に基づいて、色度信号の情報量を減らすことができるという点にある。

\begin{figure}[htbp]
    \begin{center}
        \includegraphics[bb=0 0 681 222,width=14cm]{img/yuv-pixel-structure.pdf}
    \end{center}
    \caption{YUVのピクセルあたりの色情報の構造}
    \label{fig:yuv-pixel-structure}
\end{figure}

% YUVのピクセルあたりの色情報の構造を、図\ref{fig:yuv-pixel-structure}に示す。

YUV 4:4:4では、輝度信号、色差信号共に1ピクセル毎である。
YUV 4:2:2では、輝度信号は1ピクセル毎、色差信号は2ピクセル毎であり、同じ色深度のYUV 4:4:4と比べ、帯域はおよそ2/3となる。
YUV 4:2:0では、輝度信号は1ピクセル毎、色差信号は4ピクセル毎であり、同じ色深度のYUV 4:2:2と比べ、帯域はおよそ3/4となり、同じ色深度のYUV 4:4:4と比べ、帯域はおよそ1/2となる。

なお、図\ref{fig:yuv-pixel-structure}で示した、UV成分であるCb、Crの色のサンプリング方法は、伝送方式の規格によって定められている。

\section{帯域}
\label{sec:bandwidth}
% この部分はあとで何処かに持っていく予定

帯域は解像度の他にも、インターレース、色空間、色深度により変化する。

表は次のように出力される。(表\ref{tb:video-bandwidth})
色深度は8bitとする。

また、1920x1080、
1920x1080p/60 2200x1125 148.5 SMPTE 274M

1Channel = 2200 * 1125 * 3 * 8 * 60 * 1.25/3

\begin{table}[htbp]
  \caption{解像度、フレームレート、色空間による帯域の変化}
  \label{tb:video-bandwidth}
  \begin{center}
  \begin{tabular}{c|c|c|c|c|c}
    \hline
    解像度     & フレームレート & 色空間  & ピクセルクロック & 帯域     & 同期区間を含んだ帯域 \\\hline\hline
    3840x2160 & 60p          & RGB    & 1.485MHz      & XX Gbps & YY Gbps           \\\hline
    3840x2160 & 60p          & YUV422 & 1.485MHz      & XX Gbps & YY Gbps           \\\hline
    3840x2160 & 60p          & YUV420 & 1.485MHz      & XX Gbps & YY Gbps           \\\hline
    3840x2160 & 30p          & RGB    & 1.485MHz      & XX Gbps & YY Gbps           \\\hline
    3840x2160 & 30p          & YUV422 & 1.485MHz      & XX Gbps & YY Gbps           \\\hline
    1920x1080 & 60p          & RGB    & 1.485MHz      & XX Gbps & YY Gbps           \\\hline
    1920x1080 & 60p          & YUV422 & 1.485MHz      & XX Gbps & YY Gbps           \\\hline
    1920x1080 & 60i          & RGB    & 1.485MHz      & XX Gbps & YY Gbps           \\\hline
    1920x1080 & 60i          & YUV422 & 1.485MHz      & XX Gbps & YY Gbps           \\\hline
  \end{tabular}\end{center}
\end{table}

\section{インターフェース}
\label{sec:interface}

\subsection{VGA}
VGA(Video Graphics Array)は、IBMが発表したアナログ映像信号の伝送規格、または、同社が開発したVGA表示回路用のチップのことを指す。
DE-15コネクタを使用し、赤、緑、青、垂直同期、水平同期の5つのアナログ信号で映像を伝送することができる。
DDC(VESA Display Data Channel)信号を使用することで、接続機器の対応する解像度を送信することができ、最近では1080pの映像を伝送する機器も多い。
PCでの映像出力方式として普及したが、アナログ信号であることや、音声伝送の手段が別途必要となるため、HDMIやDisplayPortなどのインターフェースに移行が進んでいる。

\subsection{DisplayPort}
DisplayPortは、VESA\footnote{ビデオ周辺機器に関する業界標準化団体}によって標準化された映像伝送規格であり、主に超解像度向けのインターフェースとして普及している。
DisplayPort 1.3からは、32.4 Gbpsのデータレートに対応し、8K映像の伝送にも対応している。

\subsection{DVI}
DVI(Digital Visual Interface)は、
Transition Minimized Differential Signaling、TMDS
VESA(Video Electronics Standards Association)によって標準化された デジタル映像信号の伝送規格。
であり、
赤、緑、青、クロックの4つのツイストペアケーブルで構成される。

\subsection{HDMI}
HDMI(High-Definition Multimedia Interface)は、映像、音声をデジタル信号で伝送する通信インターフェースの規格である。
DVIを基に、音声伝送機能や著作権保護機能を加えたものであり、物理層はDVIと同じTMDSを使用している。

HDMI 2.0では、帯域を18Gbpsに拡大し、4K@60pに対応
また、HDMI 2.0
1.4 / 2.0

HDMI 2.0からは、YCbCr 4:2:0方式によるピクセルエンコーディングの規格が追加され、1/2のデータレートで転送することが可能となった。

\begin{table}[htbp]
  \caption{HDMI 1.4と2.0での4K(3840x2160)映像の対応状況}
  \label{tb:video-bandwidth}
  \begin{center}
  \begin{tabular}{c|c|c|c}
    \hline
      フレームレート & ピクセルあたりの色深度 & HDMI 1.4 & HDMI 2.0\\\hline\hline
    \multirow{4}{*}{30Hz} &
        24bit & 対応   & 対応 \\\cline{2-4}
      & 30bit & 対応   & 対応 \\\cline{2-4}
      & 36bit & 対応   & 対応 \\\cline{2-4}
      & 48bit & 非対応 & 対応 \\\hline
    \multirow{4}{*}{60Hz} &
        24bit & 非対応 & 対応  \\\cline{2-4}
      & 30bit & 非対応 & 対応  \\\cline{2-4}
      & 36bit & 非対応 & 対応  \\\cline{2-4}
      & 48bit & 非対応 & 非対応 \\\hline
  \end{tabular}\end{center}
\end{table}

HDMI 1.4\cite{hdmi-spec-1-4}では、RGB、YCbCr 4:4:4、YCbCr 4:2:2の色空間がサポートされており、HDMI 2.0\cite{hdmi-spec-2-0}では、新たに4K解像度向けのYCbCr 4:2:0がサポートされた。

また、YCbCr方式

% すなわち、\ref{sec:colorspace}章では、色深度
\begin{figure}[htbp]
    \begin{center}
        \includegraphics[bb=0 0 531 141,width=13.926cm]{img/hdmi-spec-yuv-444.pdf}
    \end{center}
    \caption{HDMI 1.4 で定義されている YCbCr 4:4:4 方式におけるTMDSマッピング}
    \label{fig:hdmi-spec-yuv-444}
\end{figure}

\begin{figure}[htbp]
    \begin{center}
        \includegraphics[bb=0 0 531 141,width=13.926cm]{img/hdmi-spec-yuv-422.pdf}
    \end{center}
    \caption{HDMI 1.4 で定義されている YCbCr 4:2:2 方式におけるTMDSマッピング}
    \label{fig:hdmi-spec-yuv-422}
\end{figure}

\ref{sec:colorspace}節では、同じ色震度の場合、YUV 4:2:2はYUV 4:4:4と比べ2/3となると述べたが、HDMI 1.4で定義されているYCbCr 4:2:2では、1ピクセルあたりの色震度は変わらず、12bit
すなわち、HDMIでは色空間のYCbCr 4:4:4、YCbCr 4:2:2のどちらであっても帯域には影響しない。

\begin{figure}[htbp]
    \begin{center}
        \includegraphics[bb=0 0 591 306,width=15.5cm]{img/hdmi-spec-yuv-420.pdf}
    \end{center}
    \caption{HDMI 2.0 で定義されている YCbCr 4:2:0 方式におけるTMDSマッピング}
    \label{fig:hdmi-spec-yuv-420}
\end{figure}

HDMIでは、24bitの他に、30bit、36bit、48bitの色深度に対応しているが、YCbCr 4:2:0では、24bitのみの対応である。

\subsection{SDI}

SDI(Serial Digital Interface)は、SMPTE\footnote{米国映画テレビ技術者協会}によって標準化された映像伝送規格であり、主に業務機器向けの規格である。

同軸ケーブルを使用しているため、HDMIと比べて距離に対する減衰が少なく、HD-SDIでは、およそ100m遠方に伝送することができる。
BNC端子を使用することが一般的であり、抜け落ち防止のためのロック機能がるため、中継現場で活躍する。

解像度や帯域に応じて、SD-SDI、HD-SDI、3G-SDI、6G-SDI、12G-SDIなど複数の規格が定められている。
また、4K・8K映像を伝送するために、HD-SDIを4本1組で使用して伝送する事も可能である。

% \section{伝送手法}

\section{まとめ}
あ
