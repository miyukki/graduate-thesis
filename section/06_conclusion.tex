\chapter{結論}
\label{chap:conclusion}

\section{本研究のまとめ}

本研究では、IP伝送の制作現場における優位性を示すために、まず、現状の制作現場の映像伝送における必要条件と課題点を洗い出し、IP伝送でその必要条件を満たすことができ、さらに課題点をクリアすることができるかを仮説した。
4K映像をIP伝送するシステムを、ソフトウェアとハードウェアで実装し、それぞれについて仮説が満たせるかについて検証と評価を行った。

\section{今後の課題と展望}

本研究では、IP伝送による映像配信システムの設計をした。

ソフトウェアによる実装では、TCPで実装したがUDPで実装すべきである。

また、ハードウェアによる実装では、クロックを同じハードウェアで共有しているため、クロックについての考慮をしていない。
そのため、他のハードウェアでもクロック情報を共有する必要がある。

また、IP伝送のためのプロトコルがいくつか普及してきており、それに合わせた実装も行いたい。
