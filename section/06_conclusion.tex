\chapter{結論}
\label{chap:conclusion}

\section{本研究のまとめ}

本研究では、IP伝送の制作現場における優位性を示すために、現状の制作現場の映像伝送における必要条件と課題点を洗い出し、IP伝送でその必要条件を満たすことができ、さらに課題点をクリアすることができるかを証明した。
まず、4K映像をIP伝送するシステムを、ソフトウェアとハードウェアで実装し、それぞれについて仮説が満たせるかについて検証と評価を行った。

ソフトウェアによる実装では、汎用的なPCでもキャプチャボードと10GbpsのNICがあれば4K映像のIP伝送が行えることがわかった。
評価では、遅延がXms発生し、安定性が低い結果となった。

ハードウェアによる実装では、XilinxのKC705とTEDのHDMIインターフェースカードを使い、EthernetとHDMIのシステムの変換を担う回路を設計した。
評価では、遅延はXmsであり、安定性も高い結果となった。

\section{本研究の結論}

ハードウェアの実装では、最初に示した映像伝送における重要なポイントうち、画質、音節の劣化を防ぎ、遅延を一定以下にし安定させて稼働することが可能であった。
また、コスト面については、量産化されてきた映像機器や同軸ケーブルなどに比べて高価となってしまった。
これらの点をまとめると、映像伝送における重要なポイントが満たせ、IP伝送によるメリットも生まれることから、IP伝送を制作現場における優位性が立証できた。
また、映像制作現場における高解像度映像IP伝送システムが活用することが可能であった。

\section{今後の課題と展望}

本研究では、IP伝送による映像配信システムの設計をした。

ソフトウェアによる実装では、TCPで実装したがUDPで実装すべきである。

また、ハードウェアによる実装では、クロックを同じハードウェアで共有しているため、クロックについての考慮をしていない。
そのため、他のハードウェアでもクロック情報を共有する必要がある。

また、IP伝送のためのプロトコルがいくつか普及してきており、それに合わせた実装も行いたい。
