\chapter{結論}
\label{chap:conclusion}

\section{本研究のまとめ}

本研究では、拠点内の映像制作現場において、既存の同軸ケーブルなどによる映像伝送と比べてIP伝送が活用できるかについて述べた。
IPによる映像伝送は、すでに多くの製品が存在しているが、それらは拠点間での伝送を目的としている。
本研究をすすめるために、現状の映像伝送における必要条件と課題点を洗い出し、IP伝送でその必要条件を満たすことができ、さらに課題点をクリアすることができるかを証明した。

まず、映像伝送における重要な要件の1つである遅延の許容範囲を満たすために、映像制作現場においてどれほどの遅延が許容できるのかを調査した。
平均166.66msの遅延があると、許容できないという被験者が2割弱となった。この調査から、IP伝送では133.33ms以内の遅延に抑えるべきと結論づけた。

これからの映像制作現場において、IP伝送装置の性能要件としては、4K 30Pや4K 60Pへの対応、133.33ms以内の遅延が必要であるということがわかった。

次に、4K映像をIP伝送するシステムを、ソフトウェアとハードウェアで実装し、それぞれについて要件が満たせるかについて検証と評価を行った。

ソフトウェアによる実装では、汎用的なPCでもキャプチャーボードと10Gbpsに対応しているネットワークインターフェースがあれば4K映像のIP伝送が行えることがわかった。
評価では、遅延が平均で6フレーム、すなわち199.99ms発生し、映像制作現場における遅延の要件を満たすことができなかった。
遅延については、ソフトウェアがキャプチャーボードから映像データを取得するタイミングの制約やOSの処理に依存する部分が多いためである。
また、ソフトウェア要因ではなく、ハードウェアによる制約により、4K 60Pの伝送について評価することはできなかった。
その他に、遅延の評価では、計測回数により遅延フレームにばらつきがあり、安定したフレームレートでの出力ができていないことがわかった。

ハードウェアによる実装では、XilinxのKC705とTEDのHDMIインターフェースカードを使い、EthernetとHDMIのデータの変換を行う回路を設計した。
評価では、遅延は1フレーム以内であり、映像制作現場における遅延の要件を満たすことができた。また、YCbCr 4:2:0による4K 60Pの映像伝送も問題なく行えた。
しかし、トラフィックの評価では、FPGAが1秒あたり700万パケット以上も送信していることがわかった。
今回は、ダークファイバー環境でかつハードウェア同士の通信であったため問題にはならなかった。

\section{本研究の結論}

ハードウェア実装では、最初に示した映像制作現場におけるIP伝送装置の要件のうち、遅延を一定以下にし稼働することが可能であった。
また、YCbCr 4:2:0による4K 60Pでの伝送も可能であった。
これらの点から、ハードウェア実装では、映像伝送における必要な要件が満たすことができ、映像制作現場におけるIP伝送の優位性が立証できた。

ソフトウェア実装では、遅延が要件を超えてしまい、映像制作現場において課題があることがわかった。
% また、コスト面については、量産化されてきた映像機器や同軸ケーブルなどに比べて高価となってしまった。

% これらの点をまとめると、
% また、映像制作現場における高解像度映像IP伝送システムが活用することが可能であった。

\section{今後の課題と展望}

本研究では、IP伝送による映像配信システムの設計と実装を行った。

ソフトウェアによる実装では、キャプチャーボードを使用したため、キャプチャーと描画に最低でも2フレームの遅延が生じてしまう。
また、ソフトウェア制御であるため動作の安定も保証できない。
ソフトウェアでの伝送は、TCPレイヤー上で伝送した。
本来、映像伝送あれば、UDPレイヤー上で実装するべきであり、順序制御、再送制御などを考慮する既存のプロトコルを活用すべきであった。

ハードウェアによる実装では、FIFOベースでのパケットを生成するため、UDPレイヤー上の独自のプロトコルで1秒あたり700万パケットを生成していた。
これが効率の良いIP伝送方式ではないため、既存のIP伝送規格に合わせ実装したい。

また、ハードウェアによる実装では、クロックを同じハードウェアで共有しているため、クロックについての考慮をしていない。
そのため、他のハードウェアでもクロック情報をIP上で共有する必要がある。

% また、IP伝送のためのプロトコルがいくつか普及してきており、それに合わせた実装も行いたい。
