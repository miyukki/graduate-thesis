\chapter{結論}
\label{chap:conclusion}

\section{本研究のまとめ}

本研究では、映像制作現場におけるIP伝送の優位性を示すために、現状の映像伝送における必要条件と課題点を洗い出し、IP伝送でその必要条件を満たすことができ、さらに課題点をクリアすることができるかを証明した。

まず、映像伝送における重要な要件の1つである遅延の許容範囲を満たすために、映像制作現場においてどれほどの遅延が許容できるのかを調査した。
平均的にXmsの遅延があると、許容できないという被験者がN割を超えた。この調査から、IP伝送ではXms以内の遅延に抑えるべきと結論づけた。

次に、4K映像をIP伝送するシステムを、ソフトウェアとハードウェアで実装し、それぞれについて要件が満たせるかについて検証と評価を行った。

ソフトウェアによる実装では、汎用的なPCでもキャプチャボードと10GbpsのNICがあれば4K映像のIP伝送が行えることがわかった。
評価では、遅延がXms発生し、安定性が低い結果となった。

ハードウェアによる実装では、XilinxのKC705とTEDのHDMIインターフェースカードを使い、EthernetとHDMIのシステムの変換を担う回路を設計した。
評価では、遅延はXmsであり、安定性も高い結果となった。

\section{本研究の結論}

ハードウェアの実装では、最初に示した映像伝送における重要なポイントうち、画質、音節の劣化を防ぎ、遅延を一定以下にし安定させて稼働することが可能であった。
また、コスト面については、量産化されてきた映像機器や同軸ケーブルなどに比べて高価となってしまった。

これらの点をまとめると、映像伝送における重要なポイントが満たせ、IP伝送によるメリットも生まれることから、IP伝送を制作現場における優位性が立証できた。
また、映像制作現場における高解像度映像IP伝送システムが活用することが可能であった。

\newpage
\section{今後の課題と展望}

本研究では、IP伝送による映像配信システムの設計をした。

ソフトウェアによる実装では、TCPで実装したがUDPで実装すべきである。

また、ハードウェアによる実装では、クロックを同じハードウェアで共有しているため、クロックについての考慮をしていない。
そのため、他のハードウェアでもクロック情報を共有する必要がある。

また、IP伝送のためのプロトコルがいくつか普及してきており、それに合わせた実装も行いたい。
