\chapter{結論}
\label{chap:conclusion}

\section{本研究のまとめ}

本研究では、映像制作現場におけるIP伝送の優位性を示すために、現状の映像伝送における必要条件と課題点を洗い出し、IP伝送でその必要条件を満たすことができ、さらに課題点をクリアすることができるかを証明した。

まず、映像伝送における重要な要件の1つである遅延の許容範囲を満たすために、映像制作現場においてどれほどの遅延が許容できるのかを調査した。
平均的に166.66msの遅延があると、許容できないという被験者が2割弱となった。この調査から、IP伝送では166.66ms以内の遅延に抑えるべきと結論づけた。

次に、4K映像をIP伝送するシステムを、ソフトウェアとハードウェアで実装し、それぞれについて要件が満たせるかについて検証と評価を行った。

ソフトウェアによる実装では、汎用的なPCでもキャプチャーボードと10GbpsのNICがあれば4K映像のIP伝送が行えることがわかった。
評価では、遅延が平均で6フレーム、すなわち199.99ms発生し、映像制作現場における遅延の要件を満たすことができなかった。

ハードウェアによる実装では、XilinxのKC705とTEDのHDMIインターフェースカードを使い、EthernetとHDMIのデータの変換を行う回路を設計した。
評価では、遅延は1フレーム以内であり、映像制作現場における遅延の要件を満たすことができた。

\section{本研究の結論}

ハードウェアの実装では、最初に示した映像伝送における重要なポイントうち、遅延を一定以下にし稼働することが可能であった。
また、YCbCr 4:2:0方式により60Pでの伝送も可能であった。
% また、コスト面については、量産化されてきた映像機器や同軸ケーブルなどに比べて高価となってしまった。

これらの点をまとめると、映像伝送における必要な要件が満たせ、映像制作現場におけるIP伝送の優位性が立証できた。
% また、映像制作現場における高解像度映像IP伝送システムが活用することが可能であった。

\section{今後の課題と展望}

本研究では、IP伝送による映像配信システムの設計と実装を行った。

ソフトウェアによる実装では、キャプチャーボードを使用したため、キャプチャーと描画に最低でも2フレームの遅延が生じてしまう。
また、ソフトウェア制御であるため動作の安定も保証できない。
ソフトウェアでの伝送は、TCPレイヤー上で伝送した。本来、映像伝送あれば、UDPレイヤー上で実装するべきであるが、順序制御、再送制御などの実装を省くためであった。

ハードウェアによる実装では、FIFOベースでのパケットを生成するため、UDPレイヤー上の独自のプロトコルで1秒あたり700万パケットを生成していた。
これが効率の良いIP伝送方式ではないため、既存のIP伝送規格に合わせ実装したい。

また、ハードウェアによる実装では、クロックを同じハードウェアで共有しているため、クロックについての考慮をしていない。
そのため、他のハードウェアでもクロック情報を共有する必要がある。

% また、IP伝送のためのプロトコルがいくつか普及してきており、それに合わせた実装も行いたい。
