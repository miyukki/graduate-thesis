\begin{acknowledgment}

本研究を進めるにあたり、ご指導いただきました慶應義塾大学 環境情報学部教授 村井純博士、同学部教授 中村修博士、同学部准教授 Rodney D. Van Meter III 博士、同学部准教授 植原啓介博士、同学部准教授 中澤仁博士、SFC 研究所 上席所員 (訪問) 斉藤賢爾博士に感謝致します。

研究について日頃からご指導頂きました松谷健史博士、空閑洋平博士、理工学研究科開放環境科学専攻 後期博士課程 徳差雄太氏に感謝致します。
研究室に所属したばかりの頃から本研究に至るまで、特定の分野にこだわらない広い視点で何年生の時であっても妥協のない姿勢で向かい合い、絶えず多くのご指導をいただきました。
本研究を卒業論文としてまとめることができたのも両氏のおかげです。重ねて感謝申し上げます。

本研究の評価に必要な伝送装置の助言、機材を運搬していただいた一般社団法人 Mozilla Japan 工藤紀篤博士に感謝いたします。
評価に必要な伝送装置を借用させていただいた慶應義塾大学デジタルメディア・コンテンツ統合研究センターの皆様に感謝いたします。
長期の間、開発、実験用に4Kカメラなどの機器を借用させていただいた慶應義塾大学湘南藤沢メディアセンターマルチメディアサービスの皆様に感謝いたします。
実証実験を行ったORF2015では、実行委員会の皆様、ネットワーク環境を整備していただいたITCの皆様、ORF NOCの皆様、映像制作をしていただいた音像工房の皆様に感謝いたします。
本研究のハードウェア実装にあたっての資金援助をいただきました、山岸プロジェクトの山岸広太郎氏、関係者の皆様に感謝いたします。

研究室を通じた生活の中で多くの示唆を与えてくれた阿部涼介氏、沖幸太朗氏、木下舜氏、黒米祐馬氏、鈴木恒平氏、\CID{8705}橋佑允氏、原雅彦氏、細田航星氏、およびArch研究グループの皆様に感謝します。
また、徳田・村井・楠本・中村・高汐・バンミーター・植原・三次・中澤・武田 合同研究プロジェクトの皆様に感謝致します。

最後に、私の研究を支えてくれた両親をはじめとする親族、多くの友人・知人に感謝し、謝辞と致します。

\end{acknowledgment}
