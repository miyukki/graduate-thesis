\chapter{システムの設計・実装}
\label{chap:implementation}

本章では、\ref{chap:video-production}章で述べた映像制作現場におけるIP伝送装置の要件について評価するため、4K映像を非圧縮IP伝送をするシステムをソフトウェアとハードウェアで設計と実装を行い、その内容について述べる。

\section{ソフトウェアによる実験}
\label{chap:implementation}

ソフトウェアによる実装では、汎用的なPCに10Gbpsに対応しているネットワークインターフェース、4K対応キャプチャーボード Blackmagic Design Intensity Pro 4Kを用いて行った。
本実装の概要を、図~\ref{fig:software-implement-flow}に示す。

\begin{table}[htbp]
  \caption{ソフトウェアによる実装を検証したPCの構成}
  \label{tb:software-specification}
  \begin{center}
  \begin{tabular}{c||c}
    \hline
    OS  & Ubuntu 14.04 Desktop \\\hline
    CPU & Intel Core i7-4770 @ 3.40GHz \\\hline
    RAM & 8GB                  \\\hline
  \end{tabular}\end{center}
\end{table}

\begin{figure}[htbp]
  \begin{center}
    \includegraphics[bb=0 0 841 121,width=15.5cm]{img/software-implement-flow.pdf}
  \end{center}
  \caption{ソフトウェアによる実装の構成}
  \label{fig:software-implement-flow}
\end{figure}

\begin{figure}[htbp]
  \begin{center}
    \includegraphics[bb=0 0 1200 797,width=5cm]{img/DSC04110.JPG}
  \end{center}
  \caption{Blackmagic Design Intensity Pro 4K キャプチャーボード}
  \label{fig:ted-4k-fmc-card}
\end{figure}

送信プログラムでは、キャプチャーボードのデータをBlackmagic DeckLink SDK\cite{bmd-decklink-sdk}を用いて取得し、IP経由で伝送するプログラムを作成した。
受信プログラムでは、IP経由で受信したデータをLinux汎用的なメディアプレーヤーであるmplayerで再生するプログラムを作成した\cite{bmd-4k-streaming}。
今回はダークファイバー環境での想定であり、順序制御、再送制御の実装を省くため、TCPで実装を行った。

ソフトウェア全体のブロックダイアグラムを図~\ref{fig:software-whole-diagram}に示す。

\begin{figure}[htbp]
  \begin{center}
    \includegraphics[bb=0 0 498 352,width=14cm]{img/software-whole-diagram.pdf}
  \end{center}
  \caption{ソフトウェア全体のブロックダイアグラム}
  \label{fig:software-whole-diagram}
\end{figure}

送信PCでは映像データをDeckLink SDK経由で処理する。キャプチャーボードとSDKの仕様によりデータはフレーム単位でしか処理することができない。
この時点で1フレームの遅延の発生が懸念される。
受信プログラムでは、映像データをTCPパケット化して送信する。送信するためのネットワークインターフェースの制御はカーネルに依存する。

受信PCでは、ネットワークインターフェースから受信したTCPパケットを、受信プログラムで処理する。
受信した映像データをLinuxの汎用プレーヤーであるmplayerに受け渡し、画面上で映像が再生される。
VRAMの操作やビデオカードによる出力を行っていないた。この時点でもOSのタイミングで描画されるため、1フレーム程の遅延の発生が懸念される。

% ここは後ほど詳細を記述する。

% \section{考察}

% ソフトウェアによる実装ではいくつかの改善点がある。

\newpage
\section{ハードウェアによる実装}

% \subsection{実装の概要}

IP伝送装置は、Xilinx KC705 評価ボード\cite{xilinx-kc705}、HDMIインターフェースカードであるTED HDMI 2.0 FMCカード\cite{ted-hdmi4k}を使用した。
本実装の構成を図~\ref{fig:fpga-implement-flow}に示す。

\begin{figure}[htbp]
  \begin{center}
    \includegraphics[bb=0 0 841 299,width=15.5cm]{img/fpga-implement-flow.pdf}
  \end{center}
  \caption{ハードウェアによる実装の構成}
  \label{fig:fpga-implement-flow}
\end{figure}

\begin{figure}[htbp]
  \begin{center}
    \includegraphics[bb=0 0 1200 797,width=6cm]{img/DSC04111.JPG}
  \end{center}
  \caption{TED HDMI 2.0 FMCカード (TB-FMCH-HDMI4K)}
  \label{fig:ted-4k-fmc-card}
\end{figure}

今回の構成では、IP伝送装置とは別に1台のXilinx KC705 評価ボード、Quad SFP/SFP+カードを用いて、送信したIPパケットを折り返しする装置を使用した。
IPパケットを折り返しする装置について、IP伝送装置と直接の関係はないため、ここでの解説は割愛する。

本実装では、論理合成ツールとしてXilinx Vivado 2016.2を使用した。また、開発言語としてVerilog HDLを使用した。

\subsection{FPGAの回路設計}

本実装は、Xilinxが提供しているKintex-7シリーズ向けのHDMI 2.0のリファレンス実装であるxapp1287\cite{xilinx-xapp1287}をベースとしている。
Xilinxの提供するIPである10 Gigabit Ethernet Subsystem\cite{xilinx-pg157}、Video PHY Controller\cite{xilinx-pg230}、HDMI 1.4/2.0 Transmitter Subsystem\cite{xilinx-pg235}、及び、HDMI 1.4/2.0 Receiver Subsystem\cite{xilinx-pg236}、FIFO Generator\cite{xilinx-pg057}が使用されている。
本研究のために新たに実装をした箇所は、これらのIPに対してデータを受け渡しするモジュールとそのモジュールを含んだEthlogic Subsystemである。
FPGAの回路全体のブロックダイアグラムを図~\ref{fig:fpga-whole-diagram}に示す。なお、IP伝送装置に関係するモジュールのみを掲載している。

\begin{figure}[htbp]
  \begin{center}
    \includegraphics[bb=0 0 738 423,width=15.5cm]{img/fpga-whole-diagram.pdf}
  \end{center}
  \caption{FPGA回路全体のブロックダイアグラム}
  \label{fig:fpga-whole-diagram}
\end{figure}

受信側と送信側で、10 Gigabit Ethernet SubsystemとVideo Processing Subsystem間の受け渡しを行うため、表~\ref{tb:fpga-implement-modules}に示す4つのモジュールを実装した。

\begin{table}[htbp]
  \caption{10 Gigabit Ethernet Subsystem、及び、Video Processing Subsystemの接続のために実装したモジュール}
  \label{tb:fpga-implement-modules}
  \begin{center}
  \begin{tabular}{c|p{12cm}}
    \hline
    Name               & Description \\\hline\hline
    v\_axi4s\_eth2fifo.v & Ethernet Subsystemのクロックで、Ethernet Subsystemから送られてきた映像データをFIFOに書き込むモジュール \\\hline
    v\_axi4s\_fifo2eth.v & Ethernet Subsystemのクロックで、FIFOから読み込んだの映像データをEthernet Subsystemに送るモジュール \\\hline
    v\_axi4s\_vid2fifo.v & Video Processing Subsystemのクロックで、Video Processing Subsystemから送られてきた映像データをFIFOに書き込むモジュール \\\hline
    v\_axi4s\_fifo2vid.v & Video Processing Subsystemのクロックで、FIFOから読み込んだの映像データをVideo Processing Subsystemに送るモジュール \\\hline
  \end{tabular}\end{center}
\end{table}

10 Gigabit Ethernet Subsystemの基準クロックは64bit幅の設定で156.25MHzとなり、Video Processing Subsystemの基準クロックは300MHzとなる。
互いの基準クロックが異なるため、データをそのまま受け渡しすることはできない。

この問題を解決するため、読み書きで独立したクロックに対応したIndependent Clocking FIFOをFIFO Generatorで作成する。
今回のIP伝送装置の送信側で用いた、FIFOモジュールを図~\ref{fig:fpga-independent-clocking-fifo}に示す。

\begin{figure}[htbp]
  \begin{center}
    \includegraphics[bb=0 0 201 371,width=4cm]{img/fpga-independent-clocking-fifo.pdf}
  \end{center}
  \caption{Independent Clocking FIFO}
  \label{fig:fpga-independent-clocking-fifo}
\end{figure}

読み書きで独立したクロックの他に、入出力のデータ幅が異なっており、入力のデータ幅は35bit、出力のデータ幅は倍の70bitとなっている。理由は後述する。
Video Processing Subsystemは300MHzで35bitのデータを書き込み、10 Gigabit Ethernet Subsystemは156.25MHzのデータを読み込む。
10 Gigabit Ethernet Subsystemのクロックが早いため、FIFOがフル状態になることはない。
また、almost\_emptyフラグを使用しており、emptyになる1クロック前に知ることが可能である。

\newpage
これまでに解説したモジュールの組み合わせについて、送信側の接続を図~\ref{fig:fpga-video-ethernet-diagram}に示す。

\begin{figure}[htbp]
  \begin{center}
    \includegraphics[bb=0 0 911 166,width=15.5cm]{img/fpga-video-ethernet-diagram.pdf}
  \end{center}
  \caption{Video Stream to Ethernet Packet Subsystem Diagram}
  \label{fig:fpga-video-ethernet-diagram}
\end{figure}

同様に受信側の接続を図~\ref{fig:fpga-ethernet-video-diagram}に示す。

\begin{figure}[htbp]
  \begin{center}
    \includegraphics[bb=0 0 911 166,width=15.5cm]{img/fpga-ethernet-video-diagram.pdf}
  \end{center}
  \caption{Ethernet Packet to Video Stream Subsystem Diagram}
  \label{fig:fpga-ethernet-video-diagram}
\end{figure}

Video Processing Subsystemが出力するAxi4-Streamのtdataは映像データを表し、有効データ幅は32bitである。
また、表~\ref{tb:fpga-axi4-stream}に示すとおり、tlastがラインの終了、tuserがフレームの開始を表す。
FIFOに映像データだけを書き込んだ場合、ラインの終了、フレームの開始のタイミングが失われることとなる。
この問題を解決するため、FIFOにはAxi4-Streamのtdataの他に、tlast、tuser、tvalidも書き込む。

\begin{table}[htbp]
  \caption{Video Processing SubsystemのAxi4-Streamインターフェース}
  \label{tb:fpga-axi4-stream}
  \begin{center}
  \begin{tabular}{l|c|l}
    \hline
    Name   & Width     & Description \\\hline\hline
    tdata  & 3*BPC\footnotemark[3]*PPC\footnotemark[4] & Data \\\hline
    tlast  & 1         & End of line \\\hline
    tready & 1         & Ready \\\hline
    tuser  & 1         & Start of frame \\\hline
    tvalid & 1         & Valid \\\hline
  \end{tabular}\end{center}
\end{table}

\footnotetext[3]{Max Bits Per Component}
\footnotetext[4]{Pixels Per Clock}

図~\ref{fig:fpga-video-packet}に、本実装で用いたUDPデータの構造を示す。
UDPデータの映像データより前のヘッダー区間は6bytesとなっている。
これは、FPGA内部でEthernetパケットを構築していくと、データの先頭6bytesが丁度1クロックで送る64bitの区切り目となるためであり、FPGAで処理する際に効率が良い。
UDPのパケットはFIFOにデータがある間生成され続けるため、IPパケット上の長さは0としている。映像データによってはジャンボフレームとなる場合もある。

\begin{figure}[htbp]
  \begin{center}
    \includegraphics[bb=0 0 643 122,width=15.5cm]{img/fpga-video-packet.pdf}
  \end{center}
  \caption{UDPデータの構造}
  \label{fig:fpga-video-packet}
\end{figure}

各モジュールで映像データがどのように扱われるかの波形イメージを、図~\ref{fig:fpga-first-pixel-waveform}に示す。
HDMI 1.4/2.0 Receiver Subsystemから出力されるデータは、v\_axi4s\_vid2fifo.vによって、1クロック遅れてFIFOに書き込まれる。
FIFOはある程度のバッファリングが行われるため、一定クロック経過後にemptyが立ち下がり、データが読み取れる状態となる。
v\_axi4s\_fifo2eth.vによって、emptyの立ち下がりの1クロック遅れで、Ethernet、IP、UDPヘッダーの生成を行う。ヘッダーの生成中にも映像データがFIFOにたまり続ける。
ヘッダーの生成がおわる1クロック前にrd\_enを立ち上げ、FIFOのデータを読み取る。UDPデータとして映像データを書き込み、almost\_emptyの立ち下がりでrd\_enを立ち下げる。
% 10 Gigabit Ethernet Subsystem

\begin{figure}[htbp]
  \begin{center}
    % \fbox{
    \includegraphics[bb=0 0 1118 502,width=22.5cm,angle=270]{img/fpga-first-pixel-waveform.pdf}
    % }
  \end{center}
  \caption{2つのクロックによるモジュール間でのデータ伝送の波形イメージ}
  \label{fig:fpga-first-pixel-waveform}
\end{figure}

図~\ref{fig:fpga-ila-fifo-to-eth}と図~\ref{fig:fpga-ila-hdmi}は、本実装を稼働させたときのILA\footnote{Integrated Logic Analyzer}と呼ばれるFPGAの内部信号をモニターするためのツールを使った際に、HDMIの入力と出力を検証した様子である。

図~\ref{fig:fpga-ila-fifo-to-eth}は、10 Gigabit Ethernet Subsystemのクロックドメインでのデータの様子である。
v\_axi4s\_fifo2eth.vで送信用のFIFOからデータを読み取り、実際にデータを送信し、折り返し受信したデータをv\_axi4s\_eth2fifo.vで受信用のFIFOに書き込む状態である。
実際に送受信をしているため、数クロック程度の遅延があることが確認できる。

図~\ref{fig:fpga-ila-hdmi}は、Video Processing Subsystemのクロックドメインでのデータの様子である。
前述の通り、almost\_emptyの立ち下がりを合図に、パケットを生成してから映像データを書き込むまでに一定のクロックが経過するため、FIFOへの書き込みがバッファリングされる。
hdmi\_rx\_tvalidが頻繁に立ち上がりと立ち下がりを繰り返しているのに対し、hdmi\_tx\_tvalidはある程度まとまった周期で立ち上がりをしている様子が確認できる。

最後に、表~\ref{tb:fpga-post-implementation-utilization}に、論理合成後のFPGAのリソース使用状況について示す。

\begin{figure}[htbp]
  \begin{minipage}{0.6\hsize}
  \begin{center}
    \includegraphics[bb=0 0 1199 438,width=22.5cm,angle=270]{img/fpga-ila-fifo-to-eth.png}
  \end{center}
  \caption{ILAによるIP伝送時のFIFOとEthernet Subsystemのデータのダンプ}
  \label{fig:fpga-ila-fifo-to-eth}
  \end{minipage}
  \begin{minipage}{0.4\hsize}
  \begin{center}
    \includegraphics[bb=0 0 1199 246,width=22.5cm,angle=270]{img/fpga-ila-hdmi.png}
  \end{center}
  \caption{ILAによるIP伝送時のHDMI入出力のデータのダンプ}
  \label{fig:fpga-ila-hdmi}
  \end{minipage}
\end{figure}
%
% \begin{figure}[htbp]
%   \begin{center}
%     \includegraphics[bb=0 0 1199 246,width=22.5cm,angle=270]{img/fpga-ila-hdmi.png}
%   \end{center}
%   \caption{ILAによるIP伝送時のHDMI入出力のデータのダンプ}
%   \label{fig:fpga-ila-hdmi}
% \end{figure}

% \begin{figure}[htbp]
%     \begin{center}
%         \includegraphics[bb=0 0 1197 246,width=15.5cm]{img/fpga-ila-hdmi-x2.png}
%     \end{center}
%     \caption{図~\ref{fig:fpga-ila-hdmi}を拡大した}
%     \label{fig:fpga-ila-hdmi-x2}
% \end{figure}

\begin{table}[htbp]
  \caption{論理合成後のリソース使用状況}
  \label{tb:fpga-post-implementation-utilization}
  \begin{center}
  \begin{tabular}{l|r|r|r}
    \hline
    リソース  & 使用    & 全体    & 使用率   \\\hline\hline
    LUT      &  48474 & 203800 & 23.79\% \\\hline
    LUTRAM   &   4696 &  64000 &  7.34\% \\\hline
    FF       &  55768 & 407600 & 13.68\% \\\hline
    BRAM     & 310.50 &    445 & 69.78\% \\\hline
    DSP      &     23 &    840 &  2.74\% \\\hline
    IO       &     40 &    500 &  8.00\% \\\hline
    GT       &      4 &     16 & 25.00\% \\\hline
    BUFG     &     20 &     32 & 62.50\% \\\hline
    MMCM     &      3 &     10 & 30.00\% \\\hline
    PLL      &      1 &     10 & 10.00\% \\\hline
  \end{tabular}\end{center}
\end{table}

\newpage
\section{まとめ}
本章では、映像制作現場におけるIP伝送装置の要件について評価するため、4K映像を非圧縮IP伝送をするシステムをソフトウェアとハードウェアで設計と実装を行った。
ハードウェア実装とソフトウェア実装で互換性のある設計にはなっていないため、相互を組み合わせることはできない。

次章では、これらの実装を実際に評価する。
% ソフトウェアとハードウェアで同じ、その内容について述べる。
