% ■ アブストラクトの出力 ■
%	◆書式:
%		begin{jabstract}〜end{jabstract}	:日本語のアブストラクト
%		begin{eabstract}〜end{eabstract}	:英語のアブストラクト
%		※ 不要ならばコマンドごと消せば出力されない。

% 日本語のアブストラクト
\begin{jabstract}

近年、現行のハイビジョン放送を超える超高繊細な画質による、4K・8K映像の普及が世界中で加速している。
4K映像の帯域は2K映像と比較すると約4倍、現行のハイビジョン放送と比較すると約8倍にもなり、伝送方法とそのコストが課題である。
そのため、映像業界では、映像機器と比べて比較的安価なネットワークリソースを活用して伝送する、Video over IP化が進んでいる。

本研究では、汎用的なビデオインターフェースであるHDMIから、映像をIPパケット化し、非圧縮の4K映像を伝送するシステムを設計、実装し、ネットワーク活用した映像伝送システムについて、評価?を行う。% また、ネットワークスイッチによるルーティング、ブロードキャストなどについて検証を行う。

本研究では、4K映像キャプチャーボード、10Gbpsネットワークインターフェースカードを備えた汎用的なコンピューターで実装したソフトウェアと、Xilinxの7-Series FPGAボードで実装したハードウェアの両方を実装した。

評価として、4K非圧縮映像の伝送を行う既製品であるPFU QG70、実装したソフトウェア、実装したハードウェアのそれぞれにおいて遅延、重さ、XXXを計測し、XXXという結果となった。

\end{jabstract}


% 英語のアブストラクト
\begin{eabstract}

In recent years, the spread of 4K / 8K images is accelerating all over the world.
The bandwidth of 4K video is about 4 times as compared with 2K video, and it is about 8 times as compared with the current high vision broadcasting, and the transmission method and its cost are the issues.
For this reason, in the video industry, Video over IP is advancing, utilizing relatively inexpensive network resources compared to video equipment, to transmit.

In this research, we evaluate video transmission system using network using HDMI which is a general-purpose video interface, designing and implementing a system for IP packetizing video, transmitting uncompressed 4K video, and implementing it.

And we implemented both software implemented on a general-purpose computer equipped with 4K video capture board and 10 Gbps network interface card and hardware implemented with Xilinx's 7-Series FPGA board.

\end{eabstract}
